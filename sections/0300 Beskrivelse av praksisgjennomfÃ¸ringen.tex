\section{Beskrivelse av praksisgjennomføringen}

\subsection{Opplæring og oppfølging}
% Gjennom praksisen ble studenten oppfulgt av veileder, Andrea Tenti. Det var bi-ukentlige møter med veilederen fra virksomheten og resten av praktikantene i \gls{rnd}. Her fikk Andrea en oppdatering på hva som er gjennomført, hva som skal bli gjort, og hva som mangler. Dersom det var spørsmål underveis i praksisen, var det å stille spørsmål til veileder. Ettersom veilederen bare var tilstedet to ganger i uken (Tirsdager og Torsdager), ble man fort nødt til å finne en annen ansatt på bedriften til å spørre om hjelp. Dette ble Halvard Øverlien.\\
Under praksisperioden hadde jeg Andrea Tenti som veileder fra virksomheten. Oppfølgingen ble gjennomført gjennom ukentlige eller bi-ukentlige møter med veileder og de andre praktikantene i \gls{rnd}. I disse møtene oppdaterte vi veileder om status for våre oppgaver, hvilke utfordringer vi har, og hva som skal bli gjort. Ettersom veileder bare var til stedet tirsdager og torsdager, ble det nødvendig å spørre om hjelp fra andre ansatte. Halvard Øverlien var en som ble brukt mye dersom veileder ikke var tilgjenelig.

% -- Hvordan gikk det å følge planen for praksisarbeidet? --\\
\subsection{Gjennomføring av praksisplan}

% Praksisen begynte med en felles installasjon for all teknologien praktikantene skulle bruke. Som Linux (Ubuntu), Git, \gls{vsc} \gls{ide}, o.l. Etter at installasjonene ble gjennomført, ble studentene fra hver avdeling introdusert til deres oppgaver. \gls{rnd} viste en presentasjon foran alle praktikantene hos deres departement hva oppgaven deres var. Studentene fikk denne presentasjonen som hjelp for å skjønne hva oppgaven var. \\

% Praksisplanen var rimelig grei å følge ved starten av. Planen til studenten er å møte to dager i uken. Dette gikk greit på starten av praksisperioden, men i uke 42 begynner studenten å innse at han burde fullføre praksisen før eksamensperioden begynner på fullt. Studenten valgte da å møte opp flere dager i uken for å gjøre opp for 'tapte' timer. Se tabell \ref{tab:timefordeling} for timefordelingen gjennom praksisperioden for studenten. \\

Praksisperioden startet med en felles introduksjon og oppsett av nødvendige programvare og verktøy, som Linux (Ubuntu), \Gls{git} og \gls{vsc}. Etter dette fikk hver avdeling en presentasjon av sine spesifikke oppgaver. For min del innebar dette en detaljert gjennomgang av \gls{abc}-metoden og hvordan dette skulle implementeres. \\

I starten fulgte jeg praksisplanen relativt strengt, med to faste dager i uken på kontoret. Etter hvert som praksisen utviklet seg, innså jeg behovet for å jobbe mer for å bli ferdig før eksamensperioden. Dette førte til at jeg la inn ekstra dager og økte arbeidsmengden mot slutten av praksisen. Se tabell \ref{tab:timefordeling} for hvordan timene mine var fordelt gjennom praksisperioden.

\begin{table}[H]
    \centering
    \begin{tabular}{ccccccc}
        \textbf{Uke} & \textbf{Mandag} & \textbf{Tirsdag} & \textbf{Onsdag} & \textbf{Torsdag} & \textbf{Fredag} & \textbf{\# Timer}\\
        37 & - & - & - & 6 & - & \textbf{6}\\
        38 & - & - & 4 & 7 & - & \textbf{11}\\
        39 & - & - & 5 & 3,5 & - & \textbf{8,5}\\
        40 & - & - & 4 & 6 & - & \textbf{10}\\
        41 & - & - & 6 & 7 & - & \textbf{13}\\
        42 & - & - & - & 6 & 5,5 & \textbf{11,5}\\
        43 & - & - & 8 & 8 & 8 & \textbf{24}\\
        44 & 4,5 & 4 & 8 & 8 & 8 & \textbf{32,5}\\
        45 & 4 & - & - & 2,5 & - & \textbf{6,5}\\
        \textbf{Total} & \textbf{8,5} & \textbf{4} & \textbf{35} & \textbf{54} & \textbf{21,5} & \textbf{123}\\
    \end{tabular}
    \caption{Fordeling av timer gjennom praksisperioden}
    \label{tab:timefordeling}
\end{table}

% -- Hva fungerte bra og hva fungerte dårlig? --
\subsection{Hva fungerte bra og hva fungerte dårlig?}
Gjennom praksisen fikk jeg muligheten til å arbeide med reelle utfordringer som både styrket mine tekniske ferdigheter og ga verdifull innsikt i arbeidsmetoder. Jeg lærte å håndtere utfordringer i koden ved å utforske alternative løsninger, noe som gjorde meg tryggere på problemløsning. Det å kunne stille spørsmål til veileder eller andre ansatte når det oppsto problemer, var en stor fordel, og jeg opplevde alltid å bli møtt med hjelpsomhet og åpenhet. \\

En utfordring under praksisen var at veileder hadde dyp kompetanse innen matematikk, men begrenset erfaring med systemutvikling og programmering. Dette førte til at viktige verktøy som \gls{issue_board} ble nedprioritert, og arbeidet vårt manglet struktur basert på formelle systemutviklingsmetoder. I stedet hadde jeg egne \gls{issue}s og brukte de bi-ukentlige møtene som en form for \gls{sprint}. \\

Mot slutten av perioden opplevde jeg også tidspress, da jeg undervurderte hvor mange timer som måtte fullføres før eksamensperioden. Dette krevde ekstra innsats i de siste ukene, men ga meg samtidig erfaring med å håndtere korte tidsfrister.

% Gjennom praksisen ble studenten utfordret og testet med sine kunnskaper gjennom oppgaven som var gitt. Dette førte til at studenten ble mer komfortabel med koding, og fikk lære mye om alternative løsninger til studenten sin originale plan. Dersom man trengte hjelp, var det veldig lett å spørre om det og så svar med \textit{åpne armer}. \\

% Veilederen gjennom praksisen var en matematikker, så han hadde god kjennskap til matte-algoritmen som skulle implementeres, men ikke koding eller systemutvikling. Dette førte fort til at studentene ikke fulgte diverse systemutviklings-metodikker. \Gls{issue_board} var noe som fort ble glemt. Istedenfor hadde studenten egne \gls{issue}s, og tok de bi-ukentlige møtene som \gls{sprint}s. 