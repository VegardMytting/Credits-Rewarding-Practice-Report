% ----------------------------------------------------------------------------------------------------

\usepackage[norwegian]{babel}
\usepackage[a4paper,top=2cm,bottom=2cm,left=3cm,right=3cm,marginparwidth=1.75cm]{geometry}
\usepackage{amsmath}
\usepackage{float}
\usepackage{graphicx}
\usepackage[colorlinks=true, allcolors=black]{hyperref}
\usepackage{titlesec}
\usepackage[toc, acronyms]{glossaries}
\usepackage{comment}
\usepackage[
backend=biber,
style=alphabetic,
sorting=ynt
]{biblatex}
\usepackage[nottoc]{tocbibind}
\usepackage{subfig}

\addbibresource{resources.bib}

% ----------------------------------------------------------------------------------------------------

\newcommand{\logg}[3]{
\textbf{#1}

\quad #2\\

#3
}

% ----------------------------------------------------------------------------------------------------

\renewcommand{\thesection}{\arabic{section}}
\renewcommand{\thesubsection}{\thesection.\arabic{subsection}}
\setcounter{secnumdepth}{3}
\setcounter{tocdepth}{3}

% ----------------------------------------------------------------------------------------------------

\makeglossaries

\newglossaryentry{aktivitet}{
    name=aktivitet,
    description={\textit{Aktiviteter som er assosiert med produksjon, som transport, lagring, kasting av en \gls{item}} \cite{agicap_abc}}
}

\newglossaryentry{item}{
    name=vare,
    description={Produktet i produksjonen som skal bli analysert}
}

\newglossaryentry{sprint}{
    name=sprint,
    description={Tids-periode hvor man jobber med spesifikke oppgaver}
}

\newglossaryentry{issue_board}{
    name={issue board},
    description={Metode for å planlagge, organisere og visualisere arbeidsflyt for en produktfunksjon. Består av flere \gls{issue}s}
}

\newglossaryentry{issue}{
    name={issue},
    description={Planlagt implementering av en del av en produktfunksjon}
}

\newglossaryentry{git}{
    name={git},
    description={Et versjonskontrollsystem som sporer versjoner av filer}
}

\newglossaryentry{trace}{
    name={TRACE Insight},
    description={Hovedapplikasjonen til virksomheten}
}

\newacronym{ntnu}{NTNU}{Norges teknisk-naturvitenskapelige universitet}

\newacronym{abc}{ABC}{Aktivitetsbasert kostnad}

\newacronym{rnd}{RnD}{Research and Development}

\newacronym{api}{API}{Application Programming Interface}

\newacronym{vsc}{VS Code}{Visual Studio Code}

\newacronym{i18n}{i18n}{internationalization}

\newacronym{ide}{IDE}{Integrated Development Environment}

\newacronym{sql}{SQL}{Structured Query Language}