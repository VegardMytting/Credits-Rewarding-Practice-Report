\logg{Torsdag 26. September}{09:00 til 12:30 (3,5 timer)}{
Møtte opp klokken 09:00 for å fortsette å lese gjennom koden og forstå hvordan koden er satt opp. Etter litt lesing bestemte studenten seg for å eksperimentere i koden. Studenten lagde så et vindu til hvor den nye modulen av applikasjonen skal være. Dette ble gjort for å kunne forstå hvordan \textit{frontend} kode-stilen og teknologien til bedriften er.\\ 

Etter dette ble oppgaver fordelt på tvers av teamet. Studenten Vegard Arnesen Mytting fikk jobben til å skrive \textit{backend} til applikasjonen. Med andre ord, fikse backend endpointen til \textit{API}'en som skal skrives.\\

\textit{Postman} ble dermed lastet ned for å teste ut den allerede eksisterende backend endpointet. Etter alle testene ble skrevet, begynte studenten å se gjennom backend på python web applikasjonen for å kunne få mer kunnskap om hvordan java web applikasjonen kommer til å se ut.\\

Etter dette begynte studenten å skrive den nye java backend koden med å lage en \textit{model} klasse, en \textit{service} klasse og en \textit{controller} klasse.\\

Når klokken ble 12:30 måtte studenten dra for dagen.
}\\\\\\